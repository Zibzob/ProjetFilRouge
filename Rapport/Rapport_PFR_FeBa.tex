% Tout ce qui est mis derrière un « % » n'est pas vu par LaTeX
% On appelle cela des « commentaires ».  Les commentaires permettent de
% commenter son document - comme ce que je suis en train de faire
% actuellement - et de cacher du code - cf. la ligne \pagestyle.

\documentclass[a4paper]{article}

% Options possibles : 10pt, 11pt, 12pt (taille de la fonte)
%                     oneside, twoside (recto simple, recto-verso)
%                     draft, final (stade de développement)

\usepackage[utf8]{inputenc}   % LaTeX, comprends les accents !
\usepackage[T1]{fontenc}      % Police contenant les caractères français
\usepackage[francais]{babel}  % Placez ici une liste de langues, la
                              % dernière étant la langue principale

\usepackage[a4paper]{geometry}% Réduire les marges
% \pagestyle{headings}        % Pour mettre des entêtes avec les titres
                              % des sections en haut de page

\usepackage{graphicx}
\usepackage{subcaption}

\title{Projet Fil Rouge - Reinforcement Learning\\Un exemple simple
d'application}           % Les paramètres du titre : titre, auteur, date
\author{Antoine Barbosa \and Aurélien Febvre}


\date{}                       % La date n'est pas requise (la date du
                              % jour de compilation est utilisée en son
			      % absence)

\sloppy                       % Ne pas faire déborder les lignes dans la marge

\begin{document}

\maketitle                    % Faire un titre utilisant les données
                              % passées à \title, \author et \date

%%%%%%%%%%%%%%%%%%%%%%%%%%%%%%%%%%%%%%%%%%%%%%%%%%%%%%%%%%%%%%%%%%%%%%%%%%%%%%%
\begin{figure}
  \includegraphics[width=\linewidth]{bob1.jpg}
  \caption{A boat.}
  \label{fig:boat1}
\end{figure}

Figure \ref{fig:boat1} shows a boat.

%%%%%%%%%%%%%%%%%%%%%%%%%%%%%%%%%%%%%%%%%%%%%%%%%%%%%%%%%%%%%%%%%%%%%%%%%%%%%%%
\begin{figure}[h!]
  \centering
  \begin{subfigure}[b]{0.4\linewidth}
    \includegraphics[width=\linewidth]{bob1.jpg}
    \caption{Coffee.}
  \end{subfigure}
  \begin{subfigure}[b]{0.4\linewidth}
    \includegraphics[width=\linewidth]{bob2.jpg}
    \caption{More coffee.}
  \end{subfigure}
  \caption{The same cup of coffee. Two times.}
  \label{fig:coffee}
\end{figure}
%%%%%%%%%%%%%%%%%%%%%%%%%%%%%%%%%%%%%%%%%%%%%%%%%%%%%%%%%%%%%%%%%%%%%%%%%%%%%%%

\begin{abstract}
Lorem ipsum dolor sit amet, consectetuer adipiscing elit. Praesent semper orci
et purus. Nulla eu felis in lacus mollis facilisis. Maecenas porta. Vestibulum
ultricies, justo quis sodales molestie, nisi diam blandit arcu, eget egestas
mauris enim a lectus. Phasellus ac dolor in augue venenatis vulputate. Praesent
adipiscing. Aliquam adipiscing luctus ipsum. Vivamus non elit nec risus
convallis lobortis. Vestibulum ante ipsum primis in faucibus orci luctus et
ultrices posuere cubilia Curae; Nulla facilisi. Sed consequat pellentesque dui.
\end{abstract}

\tableofcontents              % Table des matières

% \part{Titre}                % Commencer une partie...
\usepackage{subcaption}
\section{Donec}               % Commencer une section, etc.

Donec nibh. Donec lorem odio, volutpat a, tempor et, interdum sed, eros.
Suspendisse vulputate, velit ac rhoncus imperdiet, risus augue iaculis velit, et
blandit libero dui quis turpis. Cras semper vulputate lectus. Morbi tempus
semper augue.

\subsection{Praesent}         % Section plus petite

Praesent adipiscing nisi id augue consectetuer ultrices. Aenean hendrerit tortor
quis magna condimentum accumsan. In a tortor. Fusce ut augue. Aenean interdum,
metus sit amet mollis tincidunt, nunc sapien viverra sapien, in convallis leo
diam sed nunc. Vivamus semper erat non leo. Vivamus ac ipsum eu velit convallis
blandit. Class aptent taciti sociosqu ad litora torquent per conubia nostra, per
inceptos himenaeos.

% \subsubsection{Titre}       % Encore plus petite

\subsection{Quisque}

Quisque dolor odio, aliquam quis, placerat sed, hendrerit eu, magna. Cras at
turpis et mi imperdiet lobortis. Nam eu massa et eros congue gravida. Sed
luctus. Nullam sit amet nunc a tellus lacinia tempor. Praesent tincidunt ligula
quis lacus. Nullam sodales, mi sed venenatis egestas, risus turpis dictum elit,
ac egestas augue eros eget erat. Cras faucibus.

% \paragraph{Titre}           % Toutes petites sections (le nom \paragraph
                              % n'est pas très bien choisi)

% \subparagraph{Titre}        % La dernière

% \appendix                   % Commençons les annexes

% \section{Titre}             % Annexe A

% \section{Titre}             % Annexe B

% \listoffigures              % Table des figures

% \listoftables               % Liste des tableaux

\end{document}

