% Tout ce qui est mis derrière un « % » n'est pas vu par LaTeX
% On appelle cela des « commentaires ».  Les commentaires permettent de
% commenter son document - comme ce que je suis en train de faire
% actuellement - et de cacher du code - cf. la ligne \pagestyle.

\documentclass[a4paper]{article}

% Options possibles : 10pt, 11pt, 12pt (taille de la fonte)
%                     oneside, twoside (recto simple, recto-verso)
%                     draft, final (stade de développement)

\usepackage[utf8]{inputenc}   % LaTeX, comprends les accents !
\usepackage[T1]{fontenc}      % Police contenant les caractères français
\usepackage[english, frenchb]{babel}  % Placez ici une liste de langues, la
                              % dernière étant la langue principale

\usepackage[a4paper]{geometry}% Réduire les marges
% \pagestyle{headings}        % Pour mettre des entêtes avec les titres
                              % des sections en haut de page

\usepackage{graphicx}
\usepackage{subcaption}

\title{Projet Fil Rouge - Reinforcement Learning\\Développement d'une IA pour
       le jeu 2048}           % Les paramètres du titre : titre, auteur, date
\author{Antoine Barbosa \and Aurélien Febvre}


\date{}                       % La date n'est pas requise (la date du
                              % jour de compilation est utilisée en son
			      % absence)

\sloppy                       % Ne pas faire déborder les lignes dans la marge

\begin{document}

\maketitle                    % Faire un titre utilisant les données
                              % passées à \title, \author et \date


\begin{abstract}
Lorem ipsum dolor sit amet, consectetuer adipiscing elit. Praesent semper orci
et purus. Nulla eu felis in lacus mollis facilisis. Maecenas porta. Vestibulum
ultricies, justo quis sodales molestie, nisi diam blandit arcu, eget egestas
mauris enim a lectus. Phasellus ac dolor in augue venenatis vulputate. Praesent
adipiscing. Aliquam adipiscing luctus ipsum. Vivamus non elit nec risus
convallis lobortis. Vestibulum ante ipsum primis in faucibus orci luctus et
ultrices posuere cubilia Curae; Nulla facilisi. Sed consequat pellentesque dui.
\end{abstract}

\tableofcontents              % Table des matières
\pagebreak

% Commencer une partie...
%%%%%%%%%%%%%%%%%%%%%%%%%%%%%%%%%%%%%%%%%%%%%%%%%%%%%%%%%%%%%%%%%%%%%%%%%%%%%%%
\part{Introduction - Etat de l'art}

Dans cette partie, nous introduisons le domaine de l'apprentissage par
renforcement ("reinforcement learning") dans ses grandes lignes, avant de
rentrer plus dans les détails dans la partie suivante.

% Commencer une section, etc.
\section{Un peu d'histoire}


% Section plus petite
\subsection{Praesent}

Praesent adipiscing nisi id augue consectetuer ultrices. Aenean hendrerit tortor

% Encore plus petite
\subsubsection{Titre}

\subsection{Quisque}

Quisque dolor odio, aliquam quis, placerat sed, hendrerit eu, magna. Cras at

% Toutes petites sections (le nom \paragraph n'est pas très bien choisi)
\paragraph{Titre}
Quisque dolor odio, aliquam quis, placerat sed, hendrerit eu, magna. Cras at
turpis et mi imperdiet lobortis. Nam eu massa et eros congue gravida. Sed

% La dernière
\subparagraph{Titre}
Quisque dolor odio, aliquam quis, placerat sed, hendrerit eu, magna. Cras at
turpis et mi imperdiet lobortis. Nam eu massa et eros congue gravida. Sed

% Commencer une section, etc.
\section{Où on est}

% Commencer une section, etc.
\section{Où on va (peut-être ?)}

\begin{figure}
  %\includegraphics[height=\linewidth]{bob1.jpg}
  \includegraphics[height=8cm]{bob1.jpg}
  \caption{A boat.}
  \label{fig:boat1}
\end{figure}

Figure \ref{fig:boat1} shows a boat.

\begin{figure}[h!]
  \centering
  \begin{subfigure}[b]{0.4\linewidth}
    \includegraphics[width=\linewidth]{bob1.jpg}
    \caption{Coffee.}
  \end{subfigure}
  \begin{subfigure}[b]{0.4\linewidth}
    \includegraphics[width=\linewidth]{bob2.jpg}
    \caption{More coffee.}
  \end{subfigure}
  \caption{The same cup of coffee. Two times.}
  \label{fig:coffee}
\end{figure}

%%%%%%%%%%%%%%%%%%%%%%%%%%%%%%%%%%%%%%%%%%%%%%%%%%%%%%%%%%%%%%%%%%%%%%%%%%%%%%%
\part{Bases théoriques}
\section{}
\subsection{Titre}
\subsubsection{Titre}
\paragraph{Titre}
\subparagraph{Titre}

%%%%%%%%%%%%%%%%%%%%%%%%%%%%%%%%%%%%%%%%%%%%%%%%%%%%%%%%%%%%%%%%%%%%%%%%%%%%%%%
\part{Application au 2048}
\section{Titre}
\subsection{Titre}
\subsubsection{Titre}
\paragraph{Titre}
\subparagraph{Titre}

%%%%%%%%%%%%%%%%%%%%%%%%%%%%%%%%%%%%%%%%%%%%%%%%%%%%%%%%%%%%%%%%%%%%%%%%%%%%%%%
\part{Résultats et conclusion}



% \appendix                   % Commençons les annexes

% \section{Titre}             % Annexe A

% \section{Titre}             % Annexe B

% \listoffigures              % Table des figures

% \listoftables               % Liste des tableaux

\end{document}

